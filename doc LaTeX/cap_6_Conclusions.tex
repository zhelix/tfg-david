% ---------------------------------------------------------------------
% ---------------------------------------------------------------------
% ---------------------------------------------------------------------

\chapter{Conclusiones}

Después de realizar la presente práctica hemos entendido los problemas que se encuentran en una red cableada donde coexisten diferentes formas de llegar a un mismo destino, entendiendo como poder solucionar esto mediante la utilización del protocolo (R)STP.

Se ha visto diferentes formas de eliminar problemas que puedan surgir en nuestra red , como la posibilidad de crear bucles mediante la conexión de los extremos de un cable de red en un mismo dispositivo.

Además, poder ofrecer una fiabilidad estructural de la jerarquía de STP, eliminando la posibilidad de que se conecte otro dispositivo con STP implementado y altere el funcionamiento deseado.

Se ha entendido el funcionamiento a nivel práctico del protocolo STP utilizando dispositivos MikroTik  hemos podido ver de forma práctica la configuración de los mismos de una forma pragmática.

Es sabido que, si no se implementa el protocolo adecuadamente, este no funcionará correctamente ya que se han realizado diversas pruebas de diferente naturaleza, para comprobar el alcance de cada parámetro sobre la configuración en conjunto.

En otro caso con mas complejidad, con un numero más elevado de nodos, se puede apreciar que aumenta dificultad y se debería elegir, por ejemplo, que prioridad y que caminos se deben elegir para optimizar el funcionamiento de la red.

Así que podemos resumir que en esta práctica hemos aprendido el funcionamiento del protocolo (R)STP sobre dispositivos MikroTik.
