% ---------------------------------------------------------------------
% ---------------------------------------------------------------------
% ---------------------------------------------------------------------

\chapter{Pruebas y resultados}

\section{Introducción}

Se han realizado una serie de pruebas para evaluar el funcionamiento del protocolo\textit{ Spanning-Tree-Protocol} (STP), y generar situaciones en las que pueden generarse problemas y malfuncionamientos.

En el \autoref{cap4} se han comentado las estructuras de red creadas y las configuraciones utilizadas en cada caso.

La configuración inicial (podemos ver la configuración en \autoref{box}), muestra los siguientes parámetros, mostrados en las figuras \autoref{fok1}, \autoref{fok2} , \autoref{fok3}, \autoref{fok4}.

\begin{figure}[h!]\centering
	\incluyeGrafico[width=1\textwidth]{fok3}
	\caption{Vista del estado y parámetros de RSTP en dispositivo \textbf{MK01}}
	\label{fok3}
	\bigskip
\end{figure}

\begin{figure}[h!]\centering
	\incluyeGrafico[width=1\textwidth]{fok4}
	\caption{Vista del estado y parámetros de RSTP en dispositivo \textbf{MK15}}
	\label{fok4}
	\bigskip
\end{figure}

\begin{figure}[h!]\centering
	\incluyeGrafico[width=1\textwidth]{fok1}
	\caption{Vista del estado y parámetros de RSTP en dispositivo \textbf{MK16}}
	\label{fok1}
	\bigskip
\end{figure}

\begin{figure}[h!]\centering
	\incluyeGrafico[width=1\textwidth]{fok2}
	\caption{Vista del estado y parámetros de RSTP en dispositivo \textbf{MK17}}
	\label{fok2}
	\bigskip
\end{figure}

\section{Desactivar el STP en modo Seguro.}

Una vez tengamos configurada de forma correcta el protocolo en todos los dispositivos y todos tengan acceso a internet, desactivaremos dentro del modo seguro de RouterOS el protocolo STP para ver cómo se altera el comportamiento normal de la red y se efectúa un bloqueo de los dispositivos al poco tiempo de su funcionamiento.

El modo seguro garantiza que si la configuración aplicada dentro de este modo cierra la sesión SAFE de forma inesperada, se cargará la ultima configuración buena conocida, la anterior con STP, de esta forma, si algún dispositivo entra en saturación por caer dentro de la redundancia de la red, se rectifica a la configuración anterior que era plenamente funcional para poder volver a acceder a cada uno de los routers.

\section{Desconectar un enlace.}

Para ver como se reestablece el cauce de datos por la red frente a la caída de un equipo o desconexión de un enlace de red, se realizará la siguiente prueba.

Con los dispositivos plenamente funcionales bajo el protocolo STP, se desconectara el cable que conecta el equipo Mikrotik1 con el Mikrotik16, donde esta conexión es la predeterminada o '\textit{root port}' para el dispositivo Mikrotik16, siendo el puerto principal para salida de este dispositivo.

Para comprobar cómo se mantiene la disponibilidad de la conexión a pesar de la desconexión o caída de un equipo colindante, se realizará un PING continuo desde el equipo conectado al MikroTik16, y acto seguido se desconectara el enlace entre los dos routers.

Si todo está correctamente configurado, una vez retiremos el cable, en el terminal donde se está realizando el PING podremos observar una perdida de paquetes o dirección inalcanzable por unos momentos, pero posteriormente, después de un tiempo de convergencia la conexión se debería de reanudar y continuar con el tráfico anterior de forma normal.


\begin{figure}[h!]\centering
	\incluyeGrafico[width=0.55\textwidth]{fdesconect4}
	\caption{Imagen del ping, devolviendo \textit{timeouts} y recuperando la conectividad}
	\label{fdesconect4}
	\bigskip
\end{figure}

\section{Modificar el valor de \textit{path cost}}

El valor Path Cost es un parámetro configurable para establecer el coste que tiene un enlace para ser utilizado. Este parámetro se utiliza en el calculo del recorrido y coste total que seguira el trafico desde el puerto de un dispositivo. De esta manera, la modificación de este parámetro permitirá redirigir el trafico en función de los costes establecidos para cada enlace y asi poder forzar que siga un camino en concreto dentro de la red.

Para probar el funcionamiento de este parametro, vamos a realizar un cambio en el valor de '\textit{Path Cost}' para modificar de esta forma el camino que tomará el trafico para enlazar con el \textit{root bridge}.

El '\textit{Path Cost}' inicial de todos los interfaces es de 10 como podemos ver en la \autoref{standard}.


\begin{figure}[h!]\centering
	\incluyeGrafico[width=0.8\textwidth]{fok1x}
	\caption{\textit{Path cost = 10}, por defecto}
	\label{standard}
	\bigskip
\end{figure}

Vemos como el puerto \textbf{'ether1}' está configurado como '\textit{root port}' i tiene un \textit{Root Path Cost} de 10.

Modificamos el valor del '\textit{Path Cost}' del puerto '\textbf{ether1}' y establecemos un '\textit{Path cost}' de 1000, como vemos en la \autoref{fdiag1cost}.

\begin{figure}[h!]\centering
	\incluyeGrafico[width=1\textwidth]{fdiag1cost}
	\caption{Cambios en \textit{path cost} a 1000.}
	\label{fdiag1cost}
	\bigskip
\end{figure}

Automáticamente se modifican los roles de los puerto donde ahora actúa como '\textit{root port}' el interfaz '\textbf{ether4}' que estaba antes como '\textit{alternated port}' debido a que ahora tiene menos coste de enlace.

\section{Proteger la integridad de la estructura cuando se añada un dispositivo en un extremo de la red configurado con STP (\textit{edge})}
\label{edge5}

El protocolo STP orientado a routing, basa su funcionamiento en la transmisión de BPDUS donde se informa a los diferentes dispositivos conectados dentro de la red de quien posee mas prioridad para dirigir el trafico. En este caso, la raíz de una configuración STP se define en el '\textit{Root Bridge}' donde un dispositivo en concreto es el que dispone de una mayor prioridad sobre los otros, definida esta por un valor en el campo \textit{\textbf{Priority}}.

El problema surge cuando dentro de una arquitectura de red, en un dispositivo del limite de la misma, se conecta otro dispositivo con el protocolo STP configurado y ademas con un valor de prioridad mas bajo que el \textit{Root} de nuestra configuración. Esto genera que se recalcule el árbol lógico de prioridades y que el nuevo dispositivo conectado redirija el trafico de la red a través de si, lo cual supone un graves problemas en seguridad e integridad de la red.

Para evitar este problema, los dispositivos que conforman el extremo de la red considerados como limite, se configuran sobre un parámetro disponible en RouterOS llamado EDGE.

Este parámetro determina que el \textit{bridge} del dispositivo se considera extremo, y a partir de el ya no se reenviaran \textit{BPDUS}, con lo cual, elimina el problema de reestructuración de la red a través de otro dispositivo externo conectado.

Para verificar este problema, se conectara un dispositivo configurado con STP y un valor de prioridad mas bajo que el valor de prioridad del root de nuestra red. Realizando la conexion del dispositivo no autorizado a un puerto de un dispositivo extremo, comprobamos que tras un tiempo de convergencia se han reestructurado todos los roles de los bridges de todos los dispositivos, encaminando el trafico ahora hacia el dispositivo nuevo, considerado ahora 'Root Bridge' por tener el menor valor de prioridad.


\section{Proteger la estructura cuando un posible usuario cree un lazo cerrado entre dos puertos (\textit{horizon})}
\label{horizon5}
Podemos evitar bucles cuando se establezca una conexión mediante un enlace físico utilizando un cable de red y conectando los dos extremos del mismo a diferentes puertos de un mismo dispositivo Mikrotik.

\begin{figure}[h!]\centering
	\incluyeGrafico[width=0.45\textwidth]{horitzon}
	\caption{Configuración valor horizon interfaz \textit{ether1}}
	\label{horitzon}
	\bigskip
\end{figure}

Configuraremos (\autoref{horizon4}) todas las interfaces del mismo dispositivo Mikrotik que conecten directamente con los clientes con el mismo valor de \textit{horizon}, como vemos en la imagen \autoref{horitzon}.

El resultado es no poder comunicarnos con otro dispositivo con el mismo valor de \textit{horizon} (\autoref{nopinghorizon}).
\begin{figure}[h!]\centering
	\incluyeGrafico[width=0.9\textwidth]{nopinghorizon}
	\caption{Dispositivos incomunicables con mismo horizon.}
	\label{nopinghorizon}
	\bigskip
\end{figure}

\section{Añadir enlace para tener redundancia (\textit{Backup Port})}

Si replicamos una conexión entre dos dispositivos previamente ya conectados, que forman parte del conjunto de dispositivos configurados como RSTP y \textbf{\textit{Edge=No}}, automáticamente el dispositivo establecerá el nuevo enlace marcándolo como \textit{backup port}.

Podemos ver el resultado de esta prueba en la \autoref{fBackupPort1}.


\begin{figure}[h!]\centering
	\incluyeGrafico[width=0.87\textwidth]{fBackupPort1}
	\caption{Configuración de \textit{backup port}}
	\label{fBackupPort1}
	\bigskip
\end{figure}






\section{Evitar los dos problemas anteriores combinados.}

Podemos combinar las soluciones propuestas anteriormente (\autoref{edge5}, \autoref{horizon5}), utilicemos la que utilicemos, vamos a obtener una série de ventajas e inconvenientes.

Si utilizamos \textbf{Edge=Yes}, obtrendremos más control sobre la integridad de la estructura STP, evitando que si se conecta un dispositivo a la red, y éste implementa STP y además tiene una prioridad menor que el dispositivo root de nuestra red, puede alterar la jerarquía previamente configurada, con esta configuración no vamos a evitar bucles en dichos puertos (\textbf{Edge=Yes}).

Para evitar los bucles en puertos (\textbf{Edge=Yes}), deberemos establecer un valor en el campo \textit{horizon}, que tiene que ser el mismo para todos los puertos que que queramos aislar, de esta forma, garantizaremos que no se podrán generar bucles, pero, no va a poder existir comunicación entre los dispositivos conectados a estos puertos.
