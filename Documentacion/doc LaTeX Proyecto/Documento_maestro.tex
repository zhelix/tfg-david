% ---------------------------------------------------------------------
%                      DOCUMENTO DE EJEMPLO
%                    QUE UTILIZA LA PLANTILLA
%                      DE LA EDITORIAL DE LA
%                UNIVERSITAT POLITÈCNICA DE VALÈNCIA
% ---------------------------------------------------------------------

\documentclass[a4, nocrop, rm, castellano]{editorialupv}

% Opciones de la clase 'editorialupv'
%
% llibre      - Libro docente
% ebookpdf    - Libro en formato PDF para e-Readers
% tesi        - Formato de tesis
% a4          - Formato A4
% rm          - Tipo de letra roman
% sf          - Tipo de letra sanserif
% crop        - Marcas de corte (cruz) 17x24 cm (+3 mm por los cuatro lados) centrada en una A4 
% nocrop     - Sin marcas de corte, respeta el tamaño 17x24 cm 
% nomathskip  - No se modifican las distancias de las ecuaciones 
%
% castellano  - La publicación está escrita en castellano
% valencia    - La publicació està escrita en valencià
% english     - La publicación está escrita en inglés

% ---------------------------------------------------------------------
% Configuraciones presonalizadas 

% ---------------------------------------------------------------------
% ---------------------------------------------------------------------
% Configuración general


\usepackage[utf8]{inputenc}
\usepackage[T1]{fontenc}




\ifEPUB
	% Soluciona error "Undefined control sequence 			\blx@shorthands" cuando backend=biber (con bibtex 		no da este error)

	\makeatletter
	\def\blx@shorthands{} 
	\makeatother 


	\ifcastellano\usepackage[spanish,es-tabla,es-sloppy]{babel}\fi
	\ifvalencia\usepackage[catalan]{babel}\fi
	\ifenglish\usepackage[english]{babel}\fi
	\newcommand{\sen}{\on{sen}}
	\date{}
\else
	\ifcastellano\usepackage[spanish,es-tabla]{babel}\fi
	\ifvalencia\usepackage[catalan]{babel}\fi
	\ifenglish\usepackage[english]{babel}						\usepackage{csquotes}\fi
	\newcommand{\sen}{\on{sen}}	
\fi


\usepackage{xcolor}
\usepackage{array,booktabs}
\usepackage{tabularx,multicol}
\usepackage{amssymb}

\ifenglish
	\raggedright % No justifica, i no divide las palabras con guiones
\fi

% ---------------------------------------------------------------------
% ---------------------------------------------------------------------
% Bibliografía

\usepackage[
	url = false,
	style = authoryear,
	hyperref = true,
	backref = true,
	backend = biber, % Otra opción es 'backend = bibtex'
	]{biblatex}

% ---------------------------------------------------------------------
% ---------------------------------------------------------------------
% Documento electrónico

\usepackage{makeidx}
\makeindex

\ifEBOOKPDF
	\colorlet{colorEnlace}{red!75!black} 
\else 
	\colorlet{colorEnlace}{black} 
\fi

\colorlet{colorEnlace}{blue} 


\usepackage[
	{colorlinks},
	{linkcolor=colorEnlace},
	{citecolor=colorEnlace},
	{urlcolor=colorEnlace},
	{bookmarksnumbered},
	{breaklinks},
	]{hyperref}

% ---------------------------------------------------------------------
% ---------------------------------------------------------------------
% Expresión de unidades según el Sistema Internacional, monedas

\usepackage{eurosym}
\usepackage{siunitx}

\ifenglish
	\sisetup{output-decimal-marker={.}}
\else
	\sisetup{output-decimal-marker={,}}
\fi

\DeclareSIUnit[number-unit-product = {\;}] \EURO{\geneuro}


% ---------------------------------------------------------------------
% ---------------------------------------------------------------------
% Comandos y entornos personalizados

\ifEPUB
	\newcommand{\incluyeGrafico}[2][]{\Picture[#2]{./figuras/#2.png width="500px"}}
%	\newcommand{\incluyeGrafico}[2][]{\Picture[#2]{./figuras/#2.png}}
\else		
	\newcommand{\incluyeGrafico}[2][]{\includegraphics[#1]{#2}}
\fi

% ---------------------------------------------------------------------

\newcommand{\ingles}[1]{\textit{#1}}

% ------------------------------------------------------------------------

\usepackage{xspace}

\newcommand{\angles}[1]{\textit{#1}\/}
\newcommand{\miUrl}[1]{{\small%
	%\texttt%
	{\underline{#1}}}}

\newcommand{\matlabr}{{\sc Matlab}$^\circledR$\xspace}
\newcommand{\simulinkr}{\textit{Simulink}$^\circledR$\xspace}
\newcommand{\matlab}{{\textsc{Matlab}}\xspace}
\newcommand{\simulink}{\textit{Simulink}\xspace}

\newcommand{\scr}{\textit{script\/}\xspace}
\newcommand{\scrs}{\textit{scripts\/}\xspace}

% ------------------------------------------------------------------------

\definecolor{griset}{rgb}{.925, .925, .925}

\ifEPUB

	\newenvironment{parrafoDestacado}
		{
		\HCode{<div class="parrafoDestacado">}
		}
		{
		\HCode{</div>}	
		}

\else

	\newsavebox{\mybox}
	\newenvironment{parrafoDestacado}
		{%
		\fboxsep = 2ex
		\fboxrule = .4pt
	  	\begin{lrbox}{\mybox}%
	  	\begin{minipage}{.85\textwidth-2\fboxsep}\itshape\parskip=2ex
		}
		{%
		\end{minipage}
	  	\end{lrbox}%
		\begin{flushright}
			\colorbox{griset}{\usebox{\mybox}}%
	  		%\fcolorbox{black}{griset}{\usebox{\mybox}}%
		\end{flushright}
		}
		
%	\newenvironment{parrafoDestacado} % Si no nos gusta el sombreado
%		{
%		\vspace{0ex}
%		\begin{quote}\noindent
%		\itshape
%		\samepage
%		\parskip=2ex
%		%\color{NavyBlue}
%		}
%		{
%		\end{quote}
%		}

\fi

% ---------------------------------------------------------------------
% ---------------------------------------------------------------------
% Símbolos matemáticos

\newcommand{\on}{\operatorname}

% ---------------------------------------------------------------------
% ---------------------------------------------------------------------
% Teoremas y ejemplos

\ifcastellano
	\newtheorem{teorema}{\upshape\bfseries Teorema}[section]
	\newtheorem{lema}{\mdseries\scshape Lema}[section]
	\newtheorem{proposicion}{\upshape\bfseries Proposición}[section]
	\newtheorem{ejemplo}{\bfseries\scshape Ejemplo}[section]
\fi

\ifvalencia % Es mantenen els mateixos noms per compatibilitat, però l'autor els pot personalitzar
	\newtheorem{teorema}{\upshape\bfseries Teorema}[section]
	\newtheorem{lema}{\mdseries\scshape Lema}[section]
	\newtheorem{proposicion}{\upshape\bfseries Proposició}[section]
	\newtheorem{ejemplo}{\bfseries\scshape Exemple}[section]
\fi

\ifenglish
	\newtheorem{teorema}{\upshape\bfseries Theorem}[section]
	\newtheorem{lema}{\mdseries\scshape Lemma}[section]
	\newtheorem{proposicion}{\upshape\bfseries Proposition}[section]
	\newtheorem{ejemplo}{\bfseries\scshape Example}[section]
\fi


% ---------------------------------------------------------------------
% Obsoleto

% En esta versión de la plantilla no es necesario utilizar el comando \ parBlanca 
% Pero lo mantenemos para compatibilidad

\newcommand
	{\parBlanca}
	{\clearpage{\thispagestyle{empty}\cleardoublepage}}


% ---------------------------------------------------------------------



%\bibliography{referencias}

% ---------------------------------------------------------------------
% Las carpetas para los gráficos

\graphicspath{
	{figuras/}
	{logos/}
	}

% ---------------------------------------------------------------------

\title{
	\textbf{Practica 1:}\\
	Ejecucion de codigo bajo \textbf{OpenCL}
	\\[3ex]
	\mdseries\large
	}

\author{
	\textbf{Rodríguez} Martinez, David\\
	\\[2.5cm]
 \incluyeGrafico[width=0.4\textwidth]{logocl}
}

% ---------------------------------------------------------------------
% ---------------------------------------------------------------------
% ---------------------------------------------------------------------

\begin{document}

% ---------------------------------------------------------------------
% Referencias cruzadas personalizadas y texto automático de LaTeX

\ifcastellano % Las definiciones predefinidas empiezan con mayúscula
	\renewcommand{\itemautorefname}{punto}
	\renewcommand{\sectionautorefname}{sección}
	\renewcommand{\subsectionautorefname}{subsección}
	\renewcommand{\subsubsectionautorefname}{subsección}
	\renewcommand{\figureautorefname}{figura}
	\renewcommand{\tableautorefname}{tabla}
	
	\renewcommand{\indexname}{Índice alfabético}
	\renewcommand{\bibname}{Bibliografía}
	\renewcommand{\contentsname}{Índice general}
	\renewcommand{\abstractname}{Resumen}	
\fi

\ifvalencia % És molt important fer aquestes definicions, si no, apareixeran en anglès
	\renewcommand{\itemautorefname}{punt}
	\renewcommand{\sectionautorefname}{secció}
	\renewcommand{\subsectionautorefname}{subsecció}
	\renewcommand{\subsubsectionautorefname}{subsecció}
	\renewcommand{\figureautorefname}{figura}
	\renewcommand{\tableautorefname}{taula}

	\renewcommand{\indexname}{Índex alfabètic}
	\renewcommand{\bibname}{Bibliografia}
	\renewcommand{\contentsname}{Índex}
	\renewcommand{\abstractname}{Resum}	
\fi

% -------------------------------------------------------

\frontmatter

% -------------------------------------------------------
% Página de título

\maketitle	

% -------------------------------------------------------
% Resumen, prólogo o prefacio

%\cleardoublepage
%\phantomsection
%\addcontentsline{toc}{chapter}{\abstractname}

% -------------------------------------------------------
% Índice: tabla de contenidos

\ifEPUB
	% Nada, es decir no se incluye el índice general, 
	% se genera en el proceso de creación del EPUB
\else
%	\cleardoublepage
	%\phantomsection
	%\addcontentsline{toc}{chapter}{\contentsname}

	\tableofcontents
\fi

% -------------------------------------------------------
% Numeración de páginas números arábicos
% Primer capítulo en página 1

\mainmatter

% -------------------------------------------------------
% -------------------------------------------------------
% -------------------------------------------------------
% Capítulos de la publicación


\chapter{Introducción a la práctica}



% ---------------------------------------------------------------------
% ---------------------------------------------------------------------
% ---------------------------------------------------------------------

\chapter{¿Qué \textit{Spanning-Tree-Protocol}?}

\section{STP (\textit{Spanning Tree Protocol})}

En comunicaciones, STP (del inglés \textit{Spanning Tree Protocol}) es un protocolo de red de nivel 2 del modelo OSI (capa de enlace de datos).

Su función es la de gestionar la presencia de bucles en topologías de red debido a la existencia de enlaces redundantes, necesarios en muchos casos para garantizar la alta disponibilidad entre los dispositivos de una arquitectura de comunicacion.

Los bucles impiden el funcionamiento normal de la red puesto que los dispositivos de interconexión de nivel de enlace de datos reenvían indefinidamente las tramas broadcast y multicast, creando así un bucle infinito que consume tanto el ancho de banda de la red como CPU de los dispositivos de enrutamiento.

Al no existir un campo TTL (tiempo de vida) en las tramas de capa 2, éstas se quedan atrapadas indefinidamente hasta que un administrador de sistemas rompa el bucle. Un router, por el contrario, sí podría evitar este tipo de reenvíos indefinidos. La solución consiste en permitir la existencia de enlaces físicos redundantes, pero creando una topología lógica libre de bucles. STP calcula una única ruta libre de bucles entre los dispositivos de la red pero manteniendo los enlaces redundantes desactivados como reserva, con el fin de activarlos en caso de fallo.

Si la configuración de STP cambia, o si un segmento en la red redundante llega a ser inalcanzable, el algoritmo reconfigura los enlaces y restablece la conectividad, activando uno de los enlaces de reserva.

El protocolo RSTP o STP, se basan en el algoritmo de caminos mínimos de Dijkstra, aplicado en el árbol lógico obtenido a partir de la configuración de red física, normalmente en estructura de malla para la alta disponibilidad, ademas, transforma una red física con forma de malla, en la que existen bucles, por una red lógica en forma de árbol (libre de bucles). Los puentes se comunican mediante mensajes de configuración llamados \textit{Bridge Protocol Data Units}(BPDU).

El protocolo establece identificadores por puente y elige el que tiene la prioridad más alta (el número más bajo de prioridad numérica), como el puente raíz (\textit{Root Bridge}), normalmente el equipo con mas carga dentro de la red. Este puente raíz establecerá el camino de menor coste para todas las redes y en caso de cambios dentro de la arquitectura de red, el protocolo RSTP recalcula una nueva topología libre de redundancia dentro de un periodo de tiempo de convergencia.

El tráfico se redigirá desde cada puerto en función de dos parámetros principales: prioridad y dirección MAC.

La prioridad de cada \textit{bridge}, se determina  en función del que tiene la prioridad mas alta después del \textit{Root Bridge}, que es el menor valor numérico después del\textit{ Root Bridge}. Ademas, se dispone de un parámetro configurable: el \textit{Span path cost}, que se puede utilizar para redirigir el trafico a través de los nodos de la configuración.

En caso de obtener el mismo coste en dos puertos, la prioridad se determinara en función del menor valor de MAC.

Para resolver el problema se utiliza el protocolo de routing, \textit{Spanning Tree Protocol} (STP) así como sus variantes.

STP se ha convertido en el protocolo preferido para prevenir bucles de \textit{layer 2} en topologías que incluyen redundancia aunque se pueden encontrar problemas que como: tiempos de convergencia demasiados altos y una configuración que puede complicarse si no se conoce bien el principio de funcionamiento.

\begin{itemize}
	\item STP (\textbf{802.1d}): impide bucles usando un \textit{"timer"}.
	\item Rapid Spanning Tree (RSTP - \textbf{802.1w}).
\end{itemize}

\section{RSTP \textit{(Rapid Spanning Tree Protocol)}}

Rapid Spanning Tree Protocol (RSTP) es un protocolo de red de la segunda capa OSI, (nivel de enlace de datos), que gestiona enlaces redundantes. Especificado en IEEE 802.1w, es una evolución del \textit{Spanning tree Protocol} (STP), reemplazándolo en la edición 2004 del 802.1d. RSTP reduce significativamente el tiempo de convergencia de la topología de la red cuando ocurre un cambio en la topología.

Se ha convertido en el protocolo preferido para prevenir bucles de capa 2 en topologías que incluyen redundancia. Además de que el 802.1w contiene mejoras, retiene compatibilidad con su antecesor 802.1D dejando algunos parámetros sin cambiar. Por ejemplo, RSTP mantiene el mismo formato de BPDU que STP sólo que cambia el campo de versión, el cual se le asigna el valor de 2.

RSTP también define el concepto de \textit{edge-port}, el cual también se menciona en STP como \textit{PortFast}, en donde el puerto se configura como tal cuando se sabe que nunca será conectado hacia otro switch de manera que pasa inmediatamente al estado de direccionamiento sin esperar los pasos intermedios del algoritmo –etapas de escucha y aprendizaje- los cuales consumen tiempo. El tipo de enlace es detectado automáticamente, pero puede ser configurado explícitamente para hacer más rápida la convergencia.

\chapter{Presentacion del Problema}

En la práctica se van a presentar los siguientes problemas:
\begin{itemize}
	\item Instalacion del entorno.
	\item Multiplicacion de matrices  (Linux)
	\item Procedimiento de desplazamiento Geometrico (OS X)
\end{itemize}

Estos ejemplos nos permitiran visualizar como funciona en diferentes sistemas operativos (aunque similar kernel) la especificacion OpenCL para ello explicare en que consiste cada uno de los problemas listados anteriormente ademas de explicar como he realizado la instalacion en los diferentes sistemas (Linux y OSx)

\section{Instalacion del entorno}

La instalacion para MacOSx ha sido relativamente sencilla pues las librerias vienen implementadas en el programa implementado por apple Xcode asi que una vez instalado desde la Appstore ya esta listo para su funcionamietno.

Sin embargo cuando nos vamos a otro sistema operativo empiezan los problemas, este ha sido el caso de linux, que aun siendo muy similar a MacOSx, la instalacion en este puede dar problemas por la falta de librerias.

Por suerte Nvidia proporciona una serie de archivos ejecutables (.run) que ayudan a la instalacion de este entorno ademas de incorportar ejemplos 

\section{Multiplicacion de matrices}

Aqui presentare el problema pesado para su ejecucion bajo un dispositivo de computacion (en mi caso una tarjeta grafica) este problema aumentara su talla y los resultados se compararan con una ejecucion de secuencial a fin de evaluar estos resultados.

El problema consiste en:

Dos matrices A y B se dicen multiplicables si el número de columnas de A coincide con el número de filas de B.

\begin{figure}
\centering
\incluyeGrafico[width=1.0\textwidth]{matriz7}
\caption{Explicacion de como se reliza una multiplicacion de matrices}
\label{fig:matrixmul}
\end{figure}



El elemento $C_{ij}$ de la matriz producto se obtiene multiplicando cada elemento de la fila i de la matriz A por cada elemento de la columna j de la matriz B y sumándolos.

En nuestro codigo tenemos que una matriz A y una matriz B se multiplican para dar a una matriz C, para cada una hay uqe especificar la memoria que consumira, ya que es uno dee los problemas en la jerarquia de memoria de OpenCL.

\section{Procedimiento de desplazamiento Geometrico}

En esta parte de la practica he incorporado un ejemplo propuesto por Apple que a mi opinion me ha parecido curioso que consiste en una aplicacion de OpenCL y OpenGL que cuya implementacion comparte memoria de ambas tecnologias.

En particular, OpenCL y OpenGL pueden compartir datos, lo que reduce los gastos generales. Por ejemplo, los objetos de OpenGL y los objetos OpenCL creados a partir de objetos de OpenGL pueden acceder a la misma memoria. Además, GLSL (OpenGL Shading Language) shaders y almendras de OpenCL pueden acceder a los datos compartidos.

Para asegurarse de OpenCL y OpenGL hay que estabalecer ciertos parametros:
\begin{itemize}
	\item Establezcer el programa para hacer todo su cálculo y la representación en la GPU.
	\item Asignar memoria para asegurarse de que los datos se comparten de manera eficiente.
\end{itemize}

Esto mejorará el rendimiento, ya que evitará tener que transferir datos entre el host y la GPU.

\begin{figure}[h!]
\centering
\includegraphics[width=0.7\linewidth]{displacement}
\caption{}
\label{fig:displacement}
\end{figure}

En la  \autoref{fig:displacement} se puede apreciar las dos partes del problema y la solucion compartida, OpenGL genera una imagen y este realiza la carga, mientras que OpenCL se encarga de la gestion de la CPU y la memoria, todo esto nos ofrece una solucion en lo cual se comparte un buffer en el que ambas partes se interconectan.
\chapter{Ejecucion y resultados}

En este capitulo realizare las pruebas y realizare comparativas con el fin de poder calcular que mejora puede ofrecer OpenCL respecto a la programacion secuencial ya que la programacion paralela siendo mas compleja da muchas ventajas en cuanto a rendimiento y optimizacion.

\section{Multiplicacion de matrices}

En primer lugar procedere a la ejecucion de la Multiplicacion de matrices proporcionada por una serie de ejemplos de Nvidia para la computacion.

Tratara de una serie de ejecuciones en los que la talla del problema ira en aumento hasta que el programa tenga problemas en la ejecucion en terminos de memoria.

Para ello he realizado un script que realizara la ejecucion del codigo secuencialmente de cada problema y como resultado nos dara el tiempo que ha tardado cada problema en resolverse, esto nos dara una grafica de valores en los que se podra obtener el rendimiento de mejora (\textit{speedup}).
Ademas he realizado pruebas en diferentes graficas de entornos muy diferentes de procesamiento como por ejemplo una grafica de un portatil el cual esta diseñado para el ahorro de energia y una grafica potente para \textit{Gamming} la cual se le ha realizado \textit{OverClock} y esta diseñada para consumir energia y poder conseguir un aumento de rendimiento.

Para este problema, el cual solo realiza los calculos y no nos devuelve la matriz que alargaria los calculos por culpa del buffer de salida,  he implementado un multiplicador de programa que aumentara su talla exponencialmente haciendo una ejecucion muy costosa en valores elevados de la tabla, si OpenCL ofrece una potencia de paralelismo elevada los resultados deberian ser muy notables respecto a la programacion sequencial.






	



% ---------------------------------------------------------------------
% ---------------------------------------------------------------------
% ---------------------------------------------------------------------

\chapter{Pruebas y resultados}

\section{Introducción}

Se han realizado una serie de pruebas para evaluar el funcionamiento del protocolo\textit{ Spanning-Tree-Protocol} (STP), y generar situaciones en las que pueden generarse problemas y malfuncionamientos.

En el \autoref{cap4} se han comentado las estructuras de red creadas y las configuraciones utilizadas en cada caso.

La configuración inicial (podemos ver la configuración en \autoref{box}), muestra los siguientes parámetros, mostrados en las figuras \autoref{fok1}, \autoref{fok2} , \autoref{fok3}, \autoref{fok4}.

\begin{figure}[h!]\centering
	\incluyeGrafico[width=1\textwidth]{fok3}
	\caption{Vista del estado y parámetros de RSTP en dispositivo \textbf{MK01}}
	\label{fok3}
	\bigskip
\end{figure}

\begin{figure}[h!]\centering
	\incluyeGrafico[width=1\textwidth]{fok4}
	\caption{Vista del estado y parámetros de RSTP en dispositivo \textbf{MK15}}
	\label{fok4}
	\bigskip
\end{figure}

\begin{figure}[h!]\centering
	\incluyeGrafico[width=1\textwidth]{fok1}
	\caption{Vista del estado y parámetros de RSTP en dispositivo \textbf{MK16}}
	\label{fok1}
	\bigskip
\end{figure}

\begin{figure}[h!]\centering
	\incluyeGrafico[width=1\textwidth]{fok2}
	\caption{Vista del estado y parámetros de RSTP en dispositivo \textbf{MK17}}
	\label{fok2}
	\bigskip
\end{figure}

\section{Desactivar el STP en modo Seguro.}

Una vez tengamos configurada de forma correcta el protocolo en todos los dispositivos y todos tengan acceso a internet, desactivaremos dentro del modo seguro de RouterOS el protocolo STP para ver cómo se altera el comportamiento normal de la red y se efectúa un bloqueo de los dispositivos al poco tiempo de su funcionamiento.

El modo seguro garantiza que si la configuración aplicada dentro de este modo cierra la sesión SAFE de forma inesperada, se cargará la ultima configuración buena conocida, la anterior con STP, de esta forma, si algún dispositivo entra en saturación por caer dentro de la redundancia de la red, se rectifica a la configuración anterior que era plenamente funcional para poder volver a acceder a cada uno de los routers.

\section{Desconectar un enlace.}

Para ver como se reestablece el cauce de datos por la red frente a la caída de un equipo o desconexión de un enlace de red, se realizará la siguiente prueba.

Con los dispositivos plenamente funcionales bajo el protocolo STP, se desconectara el cable que conecta el equipo Mikrotik1 con el Mikrotik16, donde esta conexión es la predeterminada o '\textit{root port}' para el dispositivo Mikrotik16, siendo el puerto principal para salida de este dispositivo.

Para comprobar cómo se mantiene la disponibilidad de la conexión a pesar de la desconexión o caída de un equipo colindante, se realizará un PING continuo desde el equipo conectado al MikroTik16, y acto seguido se desconectara el enlace entre los dos routers.

Si todo está correctamente configurado, una vez retiremos el cable, en el terminal donde se está realizando el PING podremos observar una perdida de paquetes o dirección inalcanzable por unos momentos, pero posteriormente, después de un tiempo de convergencia la conexión se debería de reanudar y continuar con el tráfico anterior de forma normal.


\begin{figure}[h!]\centering
	\incluyeGrafico[width=0.55\textwidth]{fdesconect4}
	\caption{Imagen del ping, devolviendo \textit{timeouts} y recuperando la conectividad}
	\label{fdesconect4}
	\bigskip
\end{figure}

\section{Modificar el valor de \textit{path cost}}

El valor Path Cost es un parámetro configurable para establecer el coste que tiene un enlace para ser utilizado. Este parámetro se utiliza en el calculo del recorrido y coste total que seguira el trafico desde el puerto de un dispositivo. De esta manera, la modificación de este parámetro permitirá redirigir el trafico en función de los costes establecidos para cada enlace y asi poder forzar que siga un camino en concreto dentro de la red.

Para probar el funcionamiento de este parametro, vamos a realizar un cambio en el valor de '\textit{Path Cost}' para modificar de esta forma el camino que tomará el trafico para enlazar con el \textit{root bridge}.

El '\textit{Path Cost}' inicial de todos los interfaces es de 10 como podemos ver en la \autoref{standard}.


\begin{figure}[h!]\centering
	\incluyeGrafico[width=0.8\textwidth]{fok1x}
	\caption{\textit{Path cost = 10}, por defecto}
	\label{standard}
	\bigskip
\end{figure}

Vemos como el puerto \textbf{'ether1}' está configurado como '\textit{root port}' i tiene un \textit{Root Path Cost} de 10.

Modificamos el valor del '\textit{Path Cost}' del puerto '\textbf{ether1}' y establecemos un '\textit{Path cost}' de 1000, como vemos en la \autoref{fdiag1cost}.

\begin{figure}[h!]\centering
	\incluyeGrafico[width=1\textwidth]{fdiag1cost}
	\caption{Cambios en \textit{path cost} a 1000.}
	\label{fdiag1cost}
	\bigskip
\end{figure}

Automáticamente se modifican los roles de los puerto donde ahora actúa como '\textit{root port}' el interfaz '\textbf{ether4}' que estaba antes como '\textit{alternated port}' debido a que ahora tiene menos coste de enlace.

\section{Proteger la integridad de la estructura cuando se añada un dispositivo en un extremo de la red configurado con STP (\textit{edge})}
\label{edge5}

El protocolo STP orientado a routing, basa su funcionamiento en la transmisión de BPDUS donde se informa a los diferentes dispositivos conectados dentro de la red de quien posee mas prioridad para dirigir el trafico. En este caso, la raíz de una configuración STP se define en el '\textit{Root Bridge}' donde un dispositivo en concreto es el que dispone de una mayor prioridad sobre los otros, definida esta por un valor en el campo \textit{\textbf{Priority}}.

El problema surge cuando dentro de una arquitectura de red, en un dispositivo del limite de la misma, se conecta otro dispositivo con el protocolo STP configurado y ademas con un valor de prioridad mas bajo que el \textit{Root} de nuestra configuración. Esto genera que se recalcule el árbol lógico de prioridades y que el nuevo dispositivo conectado redirija el trafico de la red a través de si, lo cual supone un graves problemas en seguridad e integridad de la red.

Para evitar este problema, los dispositivos que conforman el extremo de la red considerados como limite, se configuran sobre un parámetro disponible en RouterOS llamado EDGE.

Este parámetro determina que el \textit{bridge} del dispositivo se considera extremo, y a partir de el ya no se reenviaran \textit{BPDUS}, con lo cual, elimina el problema de reestructuración de la red a través de otro dispositivo externo conectado.

Para verificar este problema, se conectara un dispositivo configurado con STP y un valor de prioridad mas bajo que el valor de prioridad del root de nuestra red. Realizando la conexion del dispositivo no autorizado a un puerto de un dispositivo extremo, comprobamos que tras un tiempo de convergencia se han reestructurado todos los roles de los bridges de todos los dispositivos, encaminando el trafico ahora hacia el dispositivo nuevo, considerado ahora 'Root Bridge' por tener el menor valor de prioridad.


\section{Proteger la estructura cuando un posible usuario cree un lazo cerrado entre dos puertos (\textit{horizon})}
\label{horizon5}
Podemos evitar bucles cuando se establezca una conexión mediante un enlace físico utilizando un cable de red y conectando los dos extremos del mismo a diferentes puertos de un mismo dispositivo Mikrotik.

\begin{figure}[h!]\centering
	\incluyeGrafico[width=0.45\textwidth]{horitzon}
	\caption{Configuración valor horizon interfaz \textit{ether1}}
	\label{horitzon}
	\bigskip
\end{figure}

Configuraremos (\autoref{horizon4}) todas las interfaces del mismo dispositivo Mikrotik que conecten directamente con los clientes con el mismo valor de \textit{horizon}, como vemos en la imagen \autoref{horitzon}.

El resultado es no poder comunicarnos con otro dispositivo con el mismo valor de \textit{horizon} (\autoref{nopinghorizon}).
\begin{figure}[h!]\centering
	\incluyeGrafico[width=0.9\textwidth]{nopinghorizon}
	\caption{Dispositivos incomunicables con mismo horizon.}
	\label{nopinghorizon}
	\bigskip
\end{figure}

\section{Añadir enlace para tener redundancia (\textit{Backup Port})}

Si replicamos una conexión entre dos dispositivos previamente ya conectados, que forman parte del conjunto de dispositivos configurados como RSTP y \textbf{\textit{Edge=No}}, automáticamente el dispositivo establecerá el nuevo enlace marcándolo como \textit{backup port}.

Podemos ver el resultado de esta prueba en la \autoref{fBackupPort1}.


\begin{figure}[h!]\centering
	\incluyeGrafico[width=0.87\textwidth]{fBackupPort1}
	\caption{Configuración de \textit{backup port}}
	\label{fBackupPort1}
	\bigskip
\end{figure}






\section{Evitar los dos problemas anteriores combinados.}

Podemos combinar las soluciones propuestas anteriormente (\autoref{edge5}, \autoref{horizon5}), utilicemos la que utilicemos, vamos a obtener una série de ventajas e inconvenientes.

Si utilizamos \textbf{Edge=Yes}, obtrendremos más control sobre la integridad de la estructura STP, evitando que si se conecta un dispositivo a la red, y éste implementa STP y además tiene una prioridad menor que el dispositivo root de nuestra red, puede alterar la jerarquía previamente configurada, con esta configuración no vamos a evitar bucles en dichos puertos (\textbf{Edge=Yes}).

Para evitar los bucles en puertos (\textbf{Edge=Yes}), deberemos establecer un valor en el campo \textit{horizon}, que tiene que ser el mismo para todos los puertos que que queramos aislar, de esta forma, garantizaremos que no se podrán generar bucles, pero, no va a poder existir comunicación entre los dispositivos conectados a estos puertos.

% ---------------------------------------------------------------------
% ---------------------------------------------------------------------
% ---------------------------------------------------------------------

\chapter{Conclusiones}

Después de realizar la presente práctica hemos entendido los problemas que se encuentran en una red cableada donde coexisten diferentes formas de llegar a un mismo destino, entendiendo como poder solucionar esto mediante la utilización del protocolo (R)STP.

Se ha visto diferentes formas de eliminar problemas que puedan surgir en nuestra red , como la posibilidad de crear bucles mediante la conexión de los extremos de un cable de red en un mismo dispositivo.

Además, poder ofrecer una fiabilidad estructural de la jerarquía de STP, eliminando la posibilidad de que se conecte otro dispositivo con STP implementado y altere el funcionamiento deseado.

Se ha entendido el funcionamiento a nivel práctico del protocolo STP utilizando dispositivos MikroTik  hemos podido ver de forma práctica la configuración de los mismos de una forma pragmática.

Es sabido que, si no se implementa el protocolo adecuadamente, este no funcionará correctamente ya que se han realizado diversas pruebas de diferente naturaleza, para comprobar el alcance de cada parámetro sobre la configuración en conjunto.

En otro caso con mas complejidad, con un numero más elevado de nodos, se puede apreciar que aumenta dificultad y se debería elegir, por ejemplo, que prioridad y que caminos se deben elegir para optimizar el funcionamiento de la red.

Así que podemos resumir que en esta práctica hemos aprendido el funcionamiento del protocolo (R)STP sobre dispositivos MikroTik.


\thispagestyle{empty}
\begin{thebibliography}{2}
	
	\bibitem{motorC}
	Llinares R. (2015) 
	\emph{Documentación de Configuración, administración y gestión de redes},\\
	Departamento de Comunicaciones,\\
	Universitat Politècnica de València.\\
	
	
	\bibitem{motorC}
	\url{http://wiki.mikrotik.com/wiki/}
	\emph{MikroTik Documentation},\\
	MikroTik maintained documentation pages,\\
	Mikrotikls Ltd.\\
\end{thebibliography}

\end{document}

% -------------------------------------------------------
% -------------------------------------------------------
% -------------------------------------------------------
% Bibliografía

%\bibitemsep = 3ex
%\bibhang = 2em

%\printbibliography[heading=bibintoc,title=\bibname]

% -------------------------------------------------------
% Índice alfabético

%\cleardoublepage
%\phantomsection
%\addcontentsline{toc}{chapter}{\indexname}

%\printindex

% -------------------------------------------------------
% Fin del documento

\end{document}

% ---------------------------------------------------------------------
% ---------------------------------------------------------------------
% ---------------------------------------------------------------------
