% ---------------------------------------------------------------------
%                      DOCUMENTO DE EJEMPLO
%                    QUE UTILIZA LA PLANTILLA
%                      DE LA EDITORIAL DE LA
%                UNIVERSITAT POLITÈCNICA DE VALÈNCIA
% ---------------------------------------------------------------------

\documentclass[a4, nocrop, rm, castellano]{editorialupv}

% Opciones de la clase 'editorialupv'
%
% llibre      - Libro docente
% ebookpdf    - Libro en formato PDF para e-Readers
% tesi        - Formato de tesis
% a4          - Formato A4
% rm          - Tipo de letra roman
% sf          - Tipo de letra sanserif
% crop        - Marcas de corte (cruz) 17x24 cm (+3 mm por los cuatro lados) centrada en una A4 
% nocrop     - Sin marcas de corte, respeta el tamaño 17x24 cm 
% nomathskip  - No se modifican las distancias de las ecuaciones 
%
% castellano  - La publicación está escrita en castellano
% valencia    - La publicació està escrita en valencià
% english     - La publicación está escrita en inglés

% ---------------------------------------------------------------------
% Configuraciones presonalizadas 

% ---------------------------------------------------------------------
% ---------------------------------------------------------------------
% Configuración general


\usepackage[utf8]{inputenc}
\usepackage[T1]{fontenc}




\ifEPUB
	% Soluciona error "Undefined control sequence 			\blx@shorthands" cuando backend=biber (con bibtex 		no da este error)

	\makeatletter
	\def\blx@shorthands{} 
	\makeatother 


	\ifcastellano\usepackage[spanish,es-tabla,es-sloppy]{babel}\fi
	\ifvalencia\usepackage[catalan]{babel}\fi
	\ifenglish\usepackage[english]{babel}\fi
	\newcommand{\sen}{\on{sen}}
	\date{}
\else
	\ifcastellano\usepackage[spanish,es-tabla]{babel}\fi
	\ifvalencia\usepackage[catalan]{babel}\fi
	\ifenglish\usepackage[english]{babel}						\usepackage{csquotes}\fi
	\newcommand{\sen}{\on{sen}}	
\fi


\usepackage{xcolor}
\usepackage{array,booktabs}
\usepackage{tabularx,multicol}
\usepackage{amssymb}

\ifenglish
	\raggedright % No justifica, i no divide las palabras con guiones
\fi

% ---------------------------------------------------------------------
% ---------------------------------------------------------------------
% Bibliografía

\usepackage[
	url = false,
	style = authoryear,
	hyperref = true,
	backref = true,
	backend = biber, % Otra opción es 'backend = bibtex'
	]{biblatex}

% ---------------------------------------------------------------------
% ---------------------------------------------------------------------
% Documento electrónico

\usepackage{makeidx}
\makeindex

\ifEBOOKPDF
	\colorlet{colorEnlace}{red!75!black} 
\else 
	\colorlet{colorEnlace}{black} 
\fi

\colorlet{colorEnlace}{blue} 


\usepackage[
	{colorlinks},
	{linkcolor=colorEnlace},
	{citecolor=colorEnlace},
	{urlcolor=colorEnlace},
	{bookmarksnumbered},
	{breaklinks},
	]{hyperref}

% ---------------------------------------------------------------------
% ---------------------------------------------------------------------
% Expresión de unidades según el Sistema Internacional, monedas

\usepackage{eurosym}
\usepackage{siunitx}

\ifenglish
	\sisetup{output-decimal-marker={.}}
\else
	\sisetup{output-decimal-marker={,}}
\fi

\DeclareSIUnit[number-unit-product = {\;}] \EURO{\geneuro}


% ---------------------------------------------------------------------
% ---------------------------------------------------------------------
% Comandos y entornos personalizados

\ifEPUB
	\newcommand{\incluyeGrafico}[2][]{\Picture[#2]{./figuras/#2.png width="500px"}}
%	\newcommand{\incluyeGrafico}[2][]{\Picture[#2]{./figuras/#2.png}}
\else		
	\newcommand{\incluyeGrafico}[2][]{\includegraphics[#1]{#2}}
\fi

% ---------------------------------------------------------------------

\newcommand{\ingles}[1]{\textit{#1}}

% ------------------------------------------------------------------------

\usepackage{xspace}

\newcommand{\angles}[1]{\textit{#1}\/}
\newcommand{\miUrl}[1]{{\small%
	%\texttt%
	{\underline{#1}}}}

\newcommand{\matlabr}{{\sc Matlab}$^\circledR$\xspace}
\newcommand{\simulinkr}{\textit{Simulink}$^\circledR$\xspace}
\newcommand{\matlab}{{\textsc{Matlab}}\xspace}
\newcommand{\simulink}{\textit{Simulink}\xspace}

\newcommand{\scr}{\textit{script\/}\xspace}
\newcommand{\scrs}{\textit{scripts\/}\xspace}

% ------------------------------------------------------------------------

\definecolor{griset}{rgb}{.925, .925, .925}

\ifEPUB

	\newenvironment{parrafoDestacado}
		{
		\HCode{<div class="parrafoDestacado">}
		}
		{
		\HCode{</div>}	
		}

\else

	\newsavebox{\mybox}
	\newenvironment{parrafoDestacado}
		{%
		\fboxsep = 2ex
		\fboxrule = .4pt
	  	\begin{lrbox}{\mybox}%
	  	\begin{minipage}{.85\textwidth-2\fboxsep}\itshape\parskip=2ex
		}
		{%
		\end{minipage}
	  	\end{lrbox}%
		\begin{flushright}
			\colorbox{griset}{\usebox{\mybox}}%
	  		%\fcolorbox{black}{griset}{\usebox{\mybox}}%
		\end{flushright}
		}
		
%	\newenvironment{parrafoDestacado} % Si no nos gusta el sombreado
%		{
%		\vspace{0ex}
%		\begin{quote}\noindent
%		\itshape
%		\samepage
%		\parskip=2ex
%		%\color{NavyBlue}
%		}
%		{
%		\end{quote}
%		}

\fi

% ---------------------------------------------------------------------
% ---------------------------------------------------------------------
% Símbolos matemáticos

\newcommand{\on}{\operatorname}

% ---------------------------------------------------------------------
% ---------------------------------------------------------------------
% Teoremas y ejemplos

\ifcastellano
	\newtheorem{teorema}{\upshape\bfseries Teorema}[section]
	\newtheorem{lema}{\mdseries\scshape Lema}[section]
	\newtheorem{proposicion}{\upshape\bfseries Proposición}[section]
	\newtheorem{ejemplo}{\bfseries\scshape Ejemplo}[section]
\fi

\ifvalencia % Es mantenen els mateixos noms per compatibilitat, però l'autor els pot personalitzar
	\newtheorem{teorema}{\upshape\bfseries Teorema}[section]
	\newtheorem{lema}{\mdseries\scshape Lema}[section]
	\newtheorem{proposicion}{\upshape\bfseries Proposició}[section]
	\newtheorem{ejemplo}{\bfseries\scshape Exemple}[section]
\fi

\ifenglish
	\newtheorem{teorema}{\upshape\bfseries Theorem}[section]
	\newtheorem{lema}{\mdseries\scshape Lemma}[section]
	\newtheorem{proposicion}{\upshape\bfseries Proposition}[section]
	\newtheorem{ejemplo}{\bfseries\scshape Example}[section]
\fi


% ---------------------------------------------------------------------
% Obsoleto

% En esta versión de la plantilla no es necesario utilizar el comando \ parBlanca 
% Pero lo mantenemos para compatibilidad

\newcommand
	{\parBlanca}
	{\clearpage{\thispagestyle{empty}\cleardoublepage}}


% ---------------------------------------------------------------------



%\bibliography{referencias}

% ---------------------------------------------------------------------
% Las carpetas para los gráficos

\graphicspath{
	{figuras/}
	{logos/}
	}

% ---------------------------------------------------------------------

\title{
	\textbf{Practica 1:}\\
	Ejecucion de codigo bajo \textbf{OpenCL}
	\\[3ex]
	\mdseries\large
	}

\author{
	\textbf{Rodríguez} Martinez, David\\
	\\[2.5cm]
 \incluyeGrafico[width=0.4\textwidth]{logocl}
}

% ---------------------------------------------------------------------
% ---------------------------------------------------------------------
% ---------------------------------------------------------------------

\begin{document}

% ---------------------------------------------------------------------
% Referencias cruzadas personalizadas y texto automático de LaTeX

\ifcastellano % Las definiciones predefinidas empiezan con mayúscula
	\renewcommand{\itemautorefname}{punto}
	\renewcommand{\sectionautorefname}{sección}
	\renewcommand{\subsectionautorefname}{subsección}
	\renewcommand{\subsubsectionautorefname}{subsección}
	\renewcommand{\figureautorefname}{figura}
	\renewcommand{\tableautorefname}{tabla}
	
	\renewcommand{\indexname}{Índice alfabético}
	\renewcommand{\bibname}{Bibliografía}
	\renewcommand{\contentsname}{Índice general}
	\renewcommand{\abstractname}{Resumen}	
\fi

\ifvalencia % És molt important fer aquestes definicions, si no, apareixeran en anglès
	\renewcommand{\itemautorefname}{punt}
	\renewcommand{\sectionautorefname}{secció}
	\renewcommand{\subsectionautorefname}{subsecció}
	\renewcommand{\subsubsectionautorefname}{subsecció}
	\renewcommand{\figureautorefname}{figura}
	\renewcommand{\tableautorefname}{taula}

	\renewcommand{\indexname}{Índex alfabètic}
	\renewcommand{\bibname}{Bibliografia}
	\renewcommand{\contentsname}{Índex}
	\renewcommand{\abstractname}{Resum}	
\fi

% -------------------------------------------------------

\frontmatter

% -------------------------------------------------------
% Página de título

\maketitle	

% -------------------------------------------------------
% Resumen, prólogo o prefacio

%\cleardoublepage
%\phantomsection
%\addcontentsline{toc}{chapter}{\abstractname}

% -------------------------------------------------------
% Índice: tabla de contenidos

\ifEPUB
	% Nada, es decir no se incluye el índice general, 
	% se genera en el proceso de creación del EPUB
\else
%	\cleardoublepage
	%\phantomsection
	%\addcontentsline{toc}{chapter}{\contentsname}

	\tableofcontents
\fi

% -------------------------------------------------------
% Numeración de páginas números arábicos
% Primer capítulo en página 1

\mainmatter

% -------------------------------------------------------
% -------------------------------------------------------
% -------------------------------------------------------
% Capítulos de la publicación


\chapter{Introducción a la práctica}

Esta práctica de Lenguajes y entornos de programación paralela consiste en la explotación de la GPU con el fin de poder obtener unos resultados de ejecución de problemas pesados para poder ser comparados entre diversos dispositivos y modelos de ejecución a fin de saber cuánta mejora ofrece este sistema de procesamiento.

El cálculo acelerado es el uso de una unidad de procesamiento gráfico en combinación con una CPU para acelerar aplicaciones de empresa, ingeniería, análisis y cálculo científico.\\
Gracias a esto las GPU aceleradoras han pasado a instalarse en centros de datos energéticamente eficientes de laboratorios gubernamentales, universidades y grandes compañías de todo el mundo. Las GPUs aceleran las aplicaciones de plataformas diversas, desde automóviles hasta teléfonos móviles y tablets, drones y robots.

Para ello he realizado la práctica bajo \textit{OpenCL}, una serie de librerías aportadas por \textit{CUDA GPU Programming}.

Con esto implementaremos un código en C que incorpora este modelo de programacion paralela que se ejecutara numerosas veces mientras se le aumenta la talla del problema, además se compararán diversas tecnologías, como la explotación de un solo núcleo (programación secuencial), \textit{CUDA} y \textit{OpenCL} (programación paralela) a fin de obtener una serie de resultados en los que se aplicaran diferentes comparaciones y conclusiones.

Además OpenCL es una especificación desarrollada por Apple, asi que se realizara una prueba en un entorno que incorpora nativamente OpenCL (OS X) con el fin de ver cómo de sencillo resulta trabajar con esta especificación en un entorno preparado.

\chapter{¿Qué es \textit{OpenCL}?}



\chapter{Presentacion del Problema}

\chapter{Ejecucion y resultados}

En este capitulo realizare las pruebas y realizare comparativas con el fin de poder calcular que mejora puede ofrecer OpenCL respecto a la programacion secuencial ya que la programacion paralela siendo mas compleja da muchas ventajas en cuanto a rendimiento y optimizacion.

\section{Multiplicacion de matrices}

En primer lugar procedere a la ejecucion de la Multiplicacion de matrices proporcionada por una serie de ejemplos de Nvidia para la computacion.

Tratara de una serie de ejecuciones en los que la talla del problema ira en aumento hasta que el programa tenga problemas en la ejecucion en terminos de memoria.

Para ello he realizado un script que realizara la ejecucion del codigo secuencialmente de cada problema y como resultado nos dara el tiempo que ha tardado cada problema en resolverse, esto nos dara una grafica de valores en los que se podra obtener el rendimiento de mejora (\textit{speedup}).
Ademas he realizado pruebas en diferentes graficas de entornos muy diferentes de procesamiento como por ejemplo una grafica de un portatil el cual esta diseñado para el ahorro de energia y una grafica potente para \textit{Gamming} la cual se le ha realizado \textit{OverClock} y esta diseñada para consumir energia y poder conseguir un aumento de rendimiento.

Para este problema, el cual solo realiza los calculos y no nos devuelve la matriz que alargaria los calculos por culpa del buffer de salida,  he implementado un multiplicador de programa que aumentara su talla exponencialmente haciendo una ejecucion muy costosa en valores elevados de la tabla, si OpenCL ofrece una potencia de paralelismo elevada los resultados deberian ser muy notables respecto a la programacion sequencial.






	



% ---------------------------------------------------------------------
% ---------------------------------------------------------------------
% ---------------------------------------------------------------------

\chapter{Pruebas y resultados}








% ---------------------------------------------------------------------
% ---------------------------------------------------------------------
% ---------------------------------------------------------------------

\chapter{Conclusiones}

Después de realizar la presente práctica hemos entendido los problemas que se encuentran en una red cableada donde coexisten diferentes formas de llegar a un mismo destino, entendiendo como poder solucionar esto mediante la utilización del protocolo (R)STP.

Se ha visto diferentes formas de eliminar problemas que puedan surgir en nuestra red , como la posibilidad de crear bucles mediante la conexión de los extremos de un cable de red en un mismo dispositivo.

Además, poder ofrecer una fiabilidad estructural de la jerarquía de STP, eliminando la posibilidad de que se conecte otro dispositivo con STP implementado y altere el funcionamiento deseado.

Se ha entendido el funcionamiento a nivel práctico del protocolo STP utilizando dispositivos MikroTik  hemos podido ver de forma práctica la configuración de los mismos de una forma pragmática.

Es sabido que, si no se implementa el protocolo adecuadamente, este no funcionará correctamente ya que se han realizado diversas pruebas de diferente naturaleza, para comprobar el alcance de cada parámetro sobre la configuración en conjunto.

En otro caso con mas complejidad, con un numero más elevado de nodos, se puede apreciar que aumenta dificultad y se debería elegir, por ejemplo, que prioridad y que caminos se deben elegir para optimizar el funcionamiento de la red.

Así que podemos resumir que en esta práctica hemos aprendido el funcionamiento del protocolo (R)STP sobre dispositivos MikroTik.


\thispagestyle{empty}
\begin{thebibliography}{2}
	

\end{thebibliography}

\end{document}

% -------------------------------------------------------
% -------------------------------------------------------
% -------------------------------------------------------
% Bibliografía

%\bibitemsep = 3ex
%\bibhang = 2em

%\printbibliography[heading=bibintoc,title=\bibname]

% -------------------------------------------------------
% Índice alfabético

%\cleardoublepage
%\phantomsection
%\addcontentsline{toc}{chapter}{\indexname}

%\printindex

% -------------------------------------------------------
% Fin del documento

\end{document}

% ---------------------------------------------------------------------
% ---------------------------------------------------------------------
% ---------------------------------------------------------------------
