
\chapter{Introducción a la práctica}

Esta práctica de Lenguajes y entornos de programación paralela consiste en la explotación de la GPU con el fin de poder obtener unos resultados de ejecución de problemas pesados para poder ser comparados entre diversos dispositivos y modelos de ejecución a fin de saber cuánta mejora ofrece este sistema de procesamiento.

El cálculo acelerado es el uso de una unidad de procesamiento gráfico en combinación con una CPU para acelerar aplicaciones de empresa, ingeniería, análisis y cálculo científico.\\
Gracias a esto las GPU aceleradoras han pasado a instalarse en centros de datos energéticamente eficientes de laboratorios gubernamentales, universidades y grandes compañías de todo el mundo. Las GPUs aceleran las aplicaciones de plataformas diversas, desde automóviles hasta teléfonos móviles y tablets, drones y robots.

Para ello he realizado la práctica bajo \textit{OpenCL}, una serie de librerías aportadas por \textit{CUDA GPU Programming}.

Con esto implementaremos un código en C que incorpora este modelo de programacion paralela que se ejecutara numerosas veces mientras se le aumenta la talla del problema, además se compararán diversas tecnologías, como la explotación de un solo núcleo (programación secuencial), \textit{CUDA} y \textit{OpenCL} (programación paralela) a fin de obtener una serie de resultados en los que se aplicaran diferentes comparaciones y conclusiones.

Además OpenCL es una especificación desarrollada por Apple, asi que se realizara una prueba en un entorno que incorpora nativamente OpenCL (OS X) con el fin de ver cómo de sencillo resulta trabajar con esta especificación en un entorno preparado.
