
\chapter{Introducción}

	En la actualidad cada día aumentan el consumo de productos, la necesidad de demanda, la producción en las fábricas, el aumento del tráfico,... todo esto ha generado en el mundo el gran problema de la contaminación.
	
	\setlength{\parindent}{0ex}Con este problema he decidido en crear una plataforma para poder medir este problema de una manera sencilla y barata para que cualquier usuario pueda tener acceso a esta información ideando como una especie de plataforma para la medición de contaminación en puntos específicos y este mandar la información vía protocolo HTTP a un servidor en el que cada usuario puede acceder cualquiera para compartir sus datos.
	
	Este proyecto puede ser útil ya que la aplicación web realizada podrá servir para el uso de gestión de una \textit{SMART City}, o simplemente para recoger información y realizar estudios de contaminación de ciertos lugares ya que la aplicación podría servir esos datos en ficheros para aplicarlos a otros programas y realizar análisis, todo muy útil en el sistema de minería de datos.

\section{Objetivos}

\setlength{\parindent}{5ex}El objetivo de este proyecto es trabajar una parte de informática electrónica y comunicar esta parte con un servidor para poder trabajar esta información y poder servirla a un cliente a través de una aplicación web la cual el usuario podrá obtener y visualizar esa información.

\setlength{\parindent}{0ex}Los objetivos son los siguientes:

\begin{itemize}
\item Desarrollo de la plataforma Hardware, en este caso he decidido utilizar la Plataforma Arduino.
\item Configuración de la plataforma para que pueda enviar la información al servidor vía HTTP.
\item Configurar el servidor para que ofrezca un servicio web y a la vez reciba la información del Arduino.
\item Configurar la base de datos para que almacene la información recibida por la plataforma Hardware.
\item Crear la aplicación web que sirva esta información detallada.
\item Esta aplicación ademas ofrecerá estas funciones:
	\begin{itemize}
	\item Podrá servir información en formato \textit{.txt} o \textit{.csv} para poder trabajar con ellos.
	
	\end{itemize}
\end{itemize}


