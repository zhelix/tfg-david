\chapter{Descripción del proyecto}

Para este proyecto voy a utilizar una micro controladora Arduino DUE desatollada por \textbf{Arduino} , en esta se implementaran una serie de sensores y librerías que permitirán realizar una serie de funciones. Para enviar la información a la red, dicha controladora necesitara una Shield que le permita conectarse a la red también desatollada por \textbf{Arduino}.

Para recoger toda esta información se diseñara una aplicación que reciba las peticiones y pueda almacenarlas en una base de datos, a su vez esta aplicación web también servirá esta información al usuario de una manera detallada para su estudio.

Finalmente se le ofrece al usuario generar unos reportes en los que este podrá descargarlos para, en un futuro, poder trabajar con esa información en otros programas o plataformas.

Esta es la explicación de las tecnologías y dispositivos que voy a utilizar:

\section{Arduino DUE}

\textbf{Arduino DUE }es una placa microcontroladora basada en el Atmel SAM3X8E ARM Cortex-M3 CPU. Es el primer Arduino basado en un Nucleo microcontrolador ARM de 32 bits ya que su versión anterior (Arduino UNO), trabajaba directamente sobre un microcontrolador.. Trabaja a una frecuencia de reloj 84 MHz, siendo programable vía USB, se le puede programar sus pines utilizando el lenguaje C++ adaptado a Arduino utilizando su propio IDE de desarrollo.
Tiene 54 pines de entradas/salidas digitales, 12 pines analogicos, 4 UARTs (\textit{Universal Asynchronous Receiver-Transmitter}), 2 DAC (\textit{Digital Analogic Connection}) y 2 I2C (\textit{Inter-Integrated Circuit})

\subsection{IDE Arduino}

Es la aplicación que se utiliza para programar estas controladoras, realizando una comunicación serie entre la maquina y la controladora podremos realizar un flash de la memoria con el fin de incorporar a la memoria interna del Arduino DUE el programa que hemos diseñado para realizar las funciones. 
Para programarlo se utiliza una versión simplificada de C++ la cual realiza una configuración inicial (Setup) y un bucle infinito (Loop), una vez arrancado el programa en la controladora este se repetirá indefinidamente realizando siempre la función programada. 

El IDE de Arduino podemos descargarlo de aquí:

\url{https://www.arduino.cc/en/Main/Software}

\subsection{Shield y sensores}

La \textbf{Shield Ethernet} de Arduino permite que la controladora se pueda conectar a Internet. La Shield basada en el chip Ethernet Wiznet W5100. Este chip nos proporciona una dirección MAC y la posibilidad que utilizando una librería oficial de Arduino (Ethernet.h) nos podamos conectar a la red ya sea usando DHCP o IP estática.

Los sensores que utilizare seran los siguientes:

\begin{itemize}
	\item DHT11 - Sensor de humedad y temperatura.
	\item Parallax CO2 Gas Sensor module MQ-7.
	\item W104 - Sensor de sonido.
	\item GL55 - Sensor de Luz
	
\end{itemize}



\subsection{GPS}

Para el proyecto vamos a utilizar el GPS Neo6mv2, este GPS es bastante util ya que es barato y de bajo consumo, ideal para la propuesta del proyecto ya que lo que se interesa es el ahorro de energia.

Este GPS nos podra enviar informacion util del satelite como la posicion, la altitud, la fecha y la hora todo utilizando un formato de cadena Hexadecimal llamada NMEA DATA

\subsection{Nmea Data}

"\textit{El National Marine Electronics Association (NMEA) ha desarrollado una especificación que define la interfaz entre varias piezas de equipos electrónicos. La norma permite la electrónica de la marina marina enviar información a los ordenadores ya otros equipos marinos.}"

Los receptores GPS están incluidos en esta especificación. Muchos de los programas de Geoposicionamiento están comprendidos bajo el formato NMEA. Estos datos incluyen PVT \textit{(Posición, Velocidad, Tiempo) }generada por el receptor GPS. La idea del NMEA es enviar información llamada frase la cual es totalmente independiente de otras frases.

\section{Laravel}

Basado en \textit{Symphony} laravel es un framework Opensource que permite desarrollar aplicaciones y servicios web sirviéndose de una especie de estructura ya creada en PHP 5.

Con este entorno desarrollare la aplicación web que nos servirá para aportar  los datos y usara algunas tecnologías para ello.

\subsection{Mysql}

\subsection{Bootstrap}

\textbf{Bootstrap} es una librería  CSS para el desarrollo de vistas para aplicaciones web. Utilizado para desarrollar la interfaz de usuario en paginas web, como los botones, formularios, cabeceras...

\subsection{Chart.js}

\textbf{Chart.js} es una librería JavaScript, utilizada para generar gráficos en el la vista de las aplicaciones web, es capaz de generar gráficos dinámicos, con tooltips, animaciones, leyendas... Útil para el proyecto pues 



